\section{Response functions}\label{sec:response_functions_def}

In Sect. \ref{subsec:key_concepts} we establish the fundamental quantities used
in the price response definitions. In Sect. \ref{subsec:time_definition} we
describe the physical time scale and the trade time scale. We introduce the
price response functions used in literature in Sect. \ref{subsec:response_def}.

%%%%%%%%%%%%%%%%%%%%%%%%%%%%%%%%%%%%%%%%%%%%%%%%%%%%%%%%%%%%%%%%%%%%%%%%%%%%%%%
\subsection{Key concepts}\label{subsec:key_concepts}

In general, three categories of currency pairs are defined: majors, crosses,
and exotics. The ``major'' foreign exchange currency pairs are the major
countries that are paired with the U.S. dollar (see Table \ref{tab:majors}).
The ``crosses'' are those majors pairs that are not paired against the U.S.
dollar. Finally, the ``exotic'' pairs usually consist of a major currency
alongside a thinly traded currency or an emerging market economy currency.
The majors are the most liquid pairs, in contrast with the exotics, who can be
much more volatile.

In foreign exchange markets, orders are execute at the best available buy or
sell price. Orders often fail to result in an immediate transaction, and are
stored in a queue called the limit order book
\cite{prop_order_book,stat_prop,predictive_pow,intro_market_micro,forex_structure}. The order
book is visible for all traders and its main purpose is to ensure that all
traders have the same information on what is offered on the market. For a
detailed description of the operation of the markets, we suggest to see Ref.
\cite{my_paper_response_financial}.

At any given time there is a best (lowest) offer to sell with price
$a\left(t\right)$, and a best (highest) bid to buy with price $b\left(t\right)$
\cite{prop_order_book,subtle_nature,account_spread,limit_ord_spread,stat_theory}.
The price gap between them is called the spread
$s\left(t\right) = a\left(t\right)-b\left(t\right)$
\cite{subtle_nature,market_digest,Bouchaud_2004,account_spread,large_prices_changes,em_stylized_facts,stat_theory,teach_spread}.
Spreads are significantly positively related to price and significantly
negatively related to trading volume. Companies with more liquidity tend to
have lower spreads
\cite{components_spread_tokyo,effects_spread,account_spread,components_spread}.
Despite the foreign exchange market is often cited as the world's largest financial
market, this description fail to consider the considerable differences in trading
volume and liquidity across different currency pairs \cite{forex_microstructure}. These differences can be
directly seen in the spread. Furthermore, the bid-ask spread is directly related
with the transaction costs to the dealer \cite{teach_spread,spread_futures}.

Due to the lack of prices information in the data, we consider a basic
definition of the price given by
\cite{patterns_forex,political_forex,forex_liquidity}. The average of the best
ask and the best bid is the midpoint price, which is defined as
\cite{prop_order_book,subtle_nature,Bouchaud_2004,large_prices_changes,em_stylized_facts,stat_theory,my_paper_response_financial,teach_spread}

\begin{equation}
    m \left(t\right) = \frac{a\left(t\right) + b\left(t\right)}{2}
\end{equation}

Price changes are typically characterized as returns. If one denotes
$S\left( t\right)$ the price of an asset at time $t$, the return
$r\left(t, \tau\right)$, at time $t$ and time lag $\tau$ is simply the relative
variation of the price from $t$ to $t + \tau$
\cite{subtle_nature,empirical_facts,asynchrony_effects_corr,tick_size_impact,causes_epps_effect,non_stationarity},
\begin{equation}\label{eq:return_general}
    r^{\left(g\right)} \left(t, \tau \right) = \frac{S\left(t + \tau\right)
    - S\left(t\right)}{S\left(t\right)}
\end{equation}

We define the returns via the midpoint price as
\begin{equation}\label{eq:midpoint_price_return}
    r\left(t,\tau\right) = \frac{m\left(t+\tau\right)-m\left(t\right)}
    {m\left(t\right)}
\end{equation}
The distribution of returns is strongly non-Gaussian and its shape continuously
depends on the return period $\tau$. Small $\tau$ values have fat tails return
distributions \cite{subtle_nature}. The trade signs are defined for general
cases as
\begin{equation}\label{eq:trade_sign_general}
    \varepsilon\left(t\right)=\text{sign}\left(S\left(t\right)
    -m\left(t-\delta\right)\right)
\end{equation}
where $\delta$ is a positive time increment. Hence we have
\begin{equation}\label{eq:trade_sign_results}
    \varepsilon\left(t\right)=\left\{
    \begin{array}{cc}
    +1, & \text{If } S\left(t\right)
    \text{ is higher than the last } m\left( t \right)\\
    -1, & \text{If } S\left(t\right)
    \text{ is lower than the last } m\left( t \right)
    \end{array}\right.
\end{equation}
$\varepsilon(t) = +1$ indicates that the trade was triggered by a market order
to buy and a trade triggered by a market order to sell yields
$\varepsilon(t) = -1$
\cite{subtle_nature,Bouchaud_2004,spread_changes_affect,quant_stock_price_response,order_flow_persistent}.



%%%%%%%%%%%%%%%%%%%%%%%%%%%%%%%%%%%%%%%%%%%%%%%%%%%%%%%%%%%%%%%%%%%%%%%%%%%%%%%
\subsection{Time definition}\label{subsec:time_definition}

%%%%%%%%%%%%%%%%%%%%%%%%%%%%%%%%%%%%%%%%%%%%%%%%%%%%%%%%%%%%%%%%%%%%%%%%%%%%%%%
\subsubsection{Trade time scale}\label{subsubsec:trade_time}

%%%%%%%%%%%%%%%%%%%%%%%%%%%%%%%%%%%%%%%%%%%%%%%%%%%%%%%%%%%%%%%%%%%%%%%%%%%%%%%
\subsubsection{Physical time scale}\label{subsubsec:physical_time}

%%%%%%%%%%%%%%%%%%%%%%%%%%%%%%%%%%%%%%%%%%%%%%%%%%%%%%%%%%%%%%%%%%%%%%%%%%%%%%%
\subsection{Response function definitions}\label{subsec:response_def}