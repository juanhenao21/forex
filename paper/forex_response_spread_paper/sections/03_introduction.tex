\section{Introduction}\label{sec:introduction}

Basic description:

The term ``pip'' is commonly used in the foreign exchange market in place of the
word ‘tick’. Pips arise as a matter of convention.
A pip refers to the incremental value in the fifth non-zero digit
position from the left. Note that it is not related to the position of the
decimal point. For example, one pip in a USD/JPY value of 113.57 would be 0.01,
while one pip for EUR/USD of 1.0434 would be 0.0001 \cite{micro_eff}.


Previous works (general):

In \cite{micro_eff} they found that smaller volumes cause larger bid-ask
spreads for technical reasons to do with measurement, whereas in
\cite{eur_int_curr,eur_change_forex} claim that larger bid-ask spreads caused
smaller volumes due to trader behavior.

In \cite{curr_speculation}, they found the spreads to be between two and four
times larger for emerging market currencies than for developed country currencies.



Previous works (specific):






Explanation of our work:






Paper distribution:

The paper is organized as follows: in Sect. \ref{sec:data_set} we present our
data set of foreign exchange pairs and briefly describe the physical and trade time. We then
analyze the definition of the response functions in Sect.
\ref{sec:response_functions_def}, and compute them for the majors pairs in
Sect. \ref{sec:response_functions_imp}.
In Sect. \ref{sec:spread_impact} we show how the spread impact the values of the
response functions. Our conclusions follows in Sect. \ref{sec:conclusion}.