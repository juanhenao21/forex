\section{Introduction}\label{sec:introduction}

The foreign exchange market is the most volatile, liquid and largest of all
financial markets in the world
\cite{forex_liquidity,info_forex,intraday_forex}. Its importance in the world
economy is prominent. It affects employment, inflation, international capital
flows, among others \cite{forex_structure}. The foreign exchange market is a
decentralized market without common trading floor
\cite{forex_structure,info_forex,teach_spread}

The term ``pip'' (Price Increment Point) is commonly used in the foreign
exchange market in place of the word ‘tick’. Pips arise as a matter of
convention. A pip refers to the incremental value in the fifth non-zero digit
position from the left. Note that it is not related to the position of the
decimal point. For example, one pip in a USD/JPY value of 113.57 would be 0.01,
while one pip for EUR/USD of 1.0434 would be 0.0001
\cite{micro_eff,forex_structure}.

The foreign exchange market has attracted a lot of attention in the last 20
years. Electronic trading has changed an opaque market to a fairly transparent
with transactions costs that are a fraction of their former level. The large
amount of data that is now available to the public make possible different
kinds of analysis to this data. A lot of research is currently carry out in
different directions
\cite{curr_speculation,forex_algorithmic,forex_inefficiency,forex_structure,electronic_forex,eur_change_forex,info_forex,intraday_forex,patterns_forex,teach_spread,spread_competition,political_forex,forex_microstructure,forex_volatility,forex_liquidity}.

In Ref. \cite{micro_eff}, McGroarty et al. found that smaller volumes cause
larger bid-ask spreads for technical reasons to do with measurement, whereas in
Refs. \cite{eur_int_curr,eur_change_forex}, Hau et al. claim that larger
bid-ask spreads caused smaller volumes due to trader behavior.

In Ref. \cite{curr_speculation}, Burnside et al. found the spreads to be
between two and four times larger for emerging market currencies than for
developed country currencies. In Ref. \cite{spread_competition}, Huang et al.
found that bid-ask spreads increase when the foreign exchange market volatility
increases, and decrease when the dealer competition increases. In Ref.
\cite{electronic_forex}, Ding et al. found the Electronic Broking Services
(EBS) reduces spreads significantly, but dealers with information advantage
tend to quote relatively wider spreads. In Ref. \cite{spread_futures}, King
analyzed the foreign exchange futures market and found that the number of
transactions is negatively related with bid-ask spread, whereas volatility in
general is positively related. In Ref. \cite{intraday_forex}, Serbinenko et al.
focus in the three major market characteristics, namely efficiency, liquidity
and volatility, finding that the market is efficient in a weak form.

Due to the lack of available data in the past, very few studies exist on price
response functions in the foreign exchange market. Even if there are several
papers about foreign exchange markets, we did not found large information about
price response functions in this market. In Ref. \cite{forex_volatility},
Melvin et al. simulate their proposed model for different region foreign
exchange markets to analyze the impact of a one-standard-deviation shock using
impulse response functions. The general pattern of response was a fairly steep
drop over the first couple of days followed by a few days of gradual decline
until the response is not statistically different from zero. In Ref.
\cite{forex_liquidity}, Mancini et al. model the price impact and return
reversal to analyze liquidity. Their model predicts that more liquid assets
should exhibit narrower spreads and lower price impact.

Here, we want to discuss, based on a series of detailed empirical results
obtained on trade by trade data, that the price response functions behave
qualitatively the same in foreign exchange markets compared with correlated
financial markets. We consider different time scales and currency pairs to
compute the price response. To facilitate the reproduction of our results, the
source code for the data analysis is available in Ref. (Poner referencia).

We delve into the price response function computation in foreign exchange
markets. We perform a empirical study in different time scales and different
years. We show the similarity between the price response functions in foreign
exchange markets and correlated financial markets. Finally, we shed light on
the spread impact in the response functions for currency pairs.

The paper is organized as follows: in Sect. \ref{sec:data_set} we present our
data set of foreign exchange pairs and briefly describe the physical and trade
time. We define the time scale we will use in Sect. \ref{sec:time_scale}, and
compute the price response functions for the majors pairs in Sect.
\ref{sec:response_functions}. In Sect. \ref{sec:spread_impact} we show how
the spread impact the values of the response functions. Our conclusions follows
in Sect. \ref{sec:conclusion}.