\section{Introduction}\label{sec:introduction}

Basic description:

The foreign exchange market is the most volatile and liquid of all financial
markets in the world \cite{intraday_forex}. It is also the largest financial
market in the world \cite{info_forex,forex_liquidity}. Its importance in the world economy
is prominent. It affects employment, inflation, international capital flows,
among others \cite{forex_structure}. The foreign exchange market is a decentralized
market without common trading floor \cite{forex_structure,info_forex,teach_spread}

In the foreign exchange market, the trading day begins in the markets of
Australia and Asia. Then the markets of Europe open and finishes late in the
afternoon in New York \cite{forex_structure}. The markets does not formally
closes during the week. Thus, using the New York time as reference, the market
opens on Sunday at 18h00 and closes on Friday at 18h00 (?).

The term ``pip'' (Price Increment Point) is commonly used in the foreign
exchange market in place of the word ‘tick’. Pips arise as a matter of
convention. A pip refers to the incremental value in the fifth non-zero digit
position from the left. Note that it is not related to the position of the
decimal point. For example, one pip in a USD/JPY value of 113.57 would be 0.01,
while one pip for EUR/USD of 1.0434 would be 0.0001
\cite{micro_eff,forex_structure}.

The foreign exchange market has attracted a lot of attention in the last 20
years. Electronic trading has changed an opaque market to a fairly transparent
with transactions costs that are a fraction of their former level. The large
amount of data that is now available to the public make possible different
kinds of analysis to this data. A lot of research is currently carry out in
different directions \cite{curr_speculation,forex_algorithmic,forex_inefficiency,forex_structure,electronic_forex,eur_change_forex,info_forex,intraday_forex,patterns_forex,teach_spread,spread_competition,political_forex,forex_microstructure,forex_volatility,forex_liquidity}.


Previous works (general):

In \cite{micro_eff} they found that smaller volumes cause larger bid-ask
spreads for technical reasons to do with measurement, whereas in
\cite{eur_int_curr,eur_change_forex} claim that larger bid-ask spreads caused
smaller volumes due to trader behavior.

In \cite{curr_speculation}, they found the spreads to be between two and four
times larger for emerging market currencies than for developed country currencies.

In \cite{electronic_forex}, they found the Electronic Broking Services (EBS)
reduces spreads significantly, but dealers with information advantage tend to
quote relatively wider spreads.

In \cite{spread_competition}, they found that bid-ask spreads increase when the
foreign exchange market volatility increases, and decrease when the dealer
competition increases.

In \cite{intraday_forex}, they focus in the three major market characteristics,
namely efficiency, liquidity and volatility, finding that the market is
efficient in weak form.

In \cite{forex_inefficiency}, they investigate the dynamics of efficiency and
long memory of four major traded currencies.

In \cite{spread_futures} they analyze the foreign exchange futures market and
found that the number of transactions is negatively related with bid-ask
spread, whereas volatility in general is positively related.




Previous works (specific):

In \cite{forex_volatility}, they simulate their proposed model for different
region foreign exchange markets to analyze the
impact of a one-standard-deviation shock using impulse response functions. The
general pattern of response was a fairly steep drop over the first couple of
days followed by a few days of gradual decline until the response is not
statistically different from zero.

In \cite{forex_liquidity}, they model the price impact and return reversal to
analyze liquidity. Their model predicts that more liquid assets should exhibit
narrower spreads and lower price impact.




Explanation of our work:

Due to the lack of available data, very few studies exist on price response functions in the foreign exchange
market.
Despite the foreign exchange market is often cited as the world's largest financial
market, this description fail to consider the considerable differences in trading
volume and liquidity across different currency pairs \cite{forex_microstructure}. These differences can be
directly seen in the spread. Furthermore, the bid-ask spread is directly related
with the transaction costs to the dealer \cite{teach_spread,spread_futures}.




Paper distribution:

The paper is organized as follows: in Sect. \ref{sec:data_set} we present our
data set of foreign exchange pairs and briefly describe the physical and trade time. We then
analyze the definition of the response functions in Sect.
\ref{sec:response_functions_def}, and compute them for the majors pairs in
Sect. \ref{sec:response_functions_imp}.
In Sect. \ref{sec:spread_impact} we show how the spread impact the values of the
response functions. Our conclusions follows in Sect. \ref{sec:conclusion}.