\section{Data set and time definition}\label{sec:data_time}

In Sect. \ref{subsec:data_set} we introduce the data set used in the paper. In
Sect. \ref{subsec:time_definition} we describe the trade time scale and the
physical time scale.

%%%%%%%%%%%%%%%%%%%%%%%%%%%%%%%%%%%%%%%%%%%%%%%%%%%%%%%%%%%%%%%%%%%%%%%%%%%%%%%
\subsection{Data set}\label{subsec:data_set}

In this study, we analyze foreign exchange pairs from the foreign exchange market.

We selected the foreign exchange market because ...

% We download currency exchange rate time series from
% OANDA via an open access API.‡ Our data comprises the
% following currencies, which are all quoted in terms of euro
% (EUR) and listed in alphabetical order of their ISO cur-
% rency code: Australian dollar (AUD), Canadian dollar (CAD),
% Swiss franc (CHF), British pound (GBP), Hong Kong dollar
% (HKD), Japanese yen (JPY), Mexican peso (MXN), Nor-
% wegian krone (NOK), New Zealand dollar (NZD), Swedish
% krona (SEK), Singapore dollar (SGD), US dollar (USD), and
% South African rand (ZAR). These currencies cover 14 of
% the 20 globally most traded currencies and each of them
% accounts for a share of at least 1% of average daily turnover
% in April 2016 according to Bank for International Settle-
% ments (2016). Since OANDA does not provide sufficiently
% complete data for the remaining six currencies, we omit them
% in our analysis. This selection of currencies gives us 14 dis-
% tinct exchange rate time series when we include a dummy
% euro time series which consists of only ones. The data set
% spans from 2 January 2005 to 9 May 2017, a period of more
% than 12 years. The start date is chosen such that we can
% observe a time window as long as possible while at the same
% time maintaining the quality of the data. There is no trading
% on certain holidays as well as from Friday night to Sun-
% day night. We download exchange rates in 10-minute time
% intervals. Note that our results are independent of the time
% resolution of the underlying data set. When studying currency
% dynamics, however, a large amount of data is a prerequi-
% site, and therefore we consider observations in 10-minute time
% intervals.

In the TAQ data set, there are two data files for each stock. One gives the
list of all successive quotes. Thus, we have the best bid price, best ask
price, available volume and the time stamp accurate to the second. The other
data file is the list of all successive trades, with the traded price, traded
volume and time stamp accurate to the second. Despite the one second accuracy
of the time stamps, in both files more than one quote or trade may be recorded
in the same second.

Due to the the time stamp accuracy, it is not possible to match each trade with
the directly preceding quote. Hence, we cannot determine the trade sign by
comparing the traded price and the preceding midpoint price
\cite{Wang_2016_cross}. In this case we need to do a preprocessing of the data
to relate the midpoint prices with the trade signs in trade time scale and in
physical time scale.

To analyze the response functions, we select the majors pairs (EUR/USD, GBP/USD,
USD/JPY, AUD/USD, USD/CHF, USD/CAD and NZD/USD) for the years 2008, 2014 and
2019.

To analyze the spread impact in the response functions, we use XX foreign exchange
pairs in the years 2010, 2014 and 2019. The list of the pairs can be seen in Appendix
\ref{app:fx_pairs_spread}.

In order to avoid overnight effects and any artifact due to the opening and
closing of the foreign exchange market, we systematically discarded the first ten and the last
ten minutes of trading in a given week
\cite{Bouchaud_2004,large_prices_changes,spread_changes_affect,Wang_2016_cross}.
Therefore, we only consider trades of the same week from Sunday 17:10:00 to Friday
16:50:00 New York local time. We will refer to this interval of time as the ``market
time".


\begin{table*}[htbp]
\begin{threeparttable}
\caption{Analyzed companies.}
\begin{tabular*}{\textwidth}{c @{\extracolsep{\fill}} ccccc}
\toprule
\bf{Company} & \bf{Symbol} & \bf{Sector} & \bf{Quotes}\tnote{1} &
\bf{Trades}\tnote{2} & \bf{Spread}\tnote{3}\tabularnewline
\midrule
Alphabet Inc. & GOOG & Information Technology (IT) & $164489$ & $19029$ &
$\$0.04$\tabularnewline
Mastercard Inc. & MA & Information Technology (IT) & $98909$ & $6977$ &
$\$0.38$\tabularnewline
CME Group Inc. & CME & Financials (F) & $98188$ & $3032$ &
$\$1.08$\tabularnewline
Goldman Sachs Group Inc. & GS & Financials (F) & $160470$ & $26227$ &
$\$0.11$\tabularnewline
Transocean Ltd. & RIG & Energy (E) & $107092$ & $11641$ &
$\$0.12$\tabularnewline
Apache Corp. & APA & Energy (E) & $103074$ & $8889$ & $\$0.13$\tabularnewline
\bottomrule
\end{tabular*}
\label{tab:companies}
\begin{tablenotes}\footnotesize
\item[1] Average number of quotes from 9:40:00 to 15:50:00 New York time.
\item[2] Average number of trades from 9:40:00 to 15:50:00 New York time.
\item[3] Average spread from 9:40:00 to 15:50:00 New York time.
\end{tablenotes}
\end{threeparttable}
\end{table*}

Table \ref{tab:companies} shows the companies analyzed with their corresponding
symbol and sector. The highest average number of quotes per day and the most
liquid stock on average from our selection for the year 2008 is Alphabet Inc.
The most traded stock on average from the group was Goldman Sachs Group Inc.
On the other side, the stock with the less quotes, less traded and less
liquidity on average for the analyzed year was CME Group Inc.

%%%%%%%%%%%%%%%%%%%%%%%%%%%%%%%%%%%%%%%%%%%%%%%%%%%%%%%%%%%%%%%%%%%%%%%%%%%%%%%
\subsection{Time definition}\label{subsec:time_definition}

A key concept in the analysis of the response functions is the time. Due to the
nature of the data, they are several options to define it.

In general, the time series are indexed in calendar time (hours, minutes,
seconds, milliseconds). Moreover, tick-by-tick data available on financial
markets all over the world is time stamped up to the millisecond, but the order
of magnitude of the guaranteed precision is much larger, usually one second or
a few hundreds of milliseconds \cite{empirical_facts}. In several papers are
used different time definitions (calendar time, physical time, event time,
trade time, tick time) \cite{empirical_facts,sampling_returns,market_making}.
The TAQ data used in the analysis has the characteristic that the trades and
quotes can not be directly related due to the time stamp resolution of only one
second \cite{Wang_2016_cross}. Hence, it is impossible to match each trade with
the directly preceding quote. However, using a classification for the trade
signs, we can compute trade signs in two scales: trade time scale and physical
time scale.

The trade time scale is increased by one unit each time a transaction happens.
The advantage of this count is that limit orders far away in the order book do
not increase the time by one unit. The main outcome of trade time scale is its
``smoothing" of data and the aggregational normality \cite{empirical_facts}.

The physical time scale is increased by one unit each time a second passes.
This means that computing the responses in this scale involves sampling
\cite{sampling_returns,Wang_2016_cross}, which has to be done carefully when
dealing for example with several stocks with different liquidity. This sampling
is made in the trade signs and in the midpoint prices.

Facing the impossibility to relate midpoint prices and trade signs with the TAQ
data in trade time scale, we will use the midpoint price of the previous second
with all the trade signs of the current second. This will be our definition of
trade time scale analysis for the response function analysis.

For physical time scale, as we can sampling, we relate the unique value of
midpoint price of a previous second with the unique trade sign value of the
current second.

Thus, trade sign values will be used in trade time scale and physical time
scale and returns will be only used in physical time scale.

%%%%%%%%%%%%%%%%%%%%%%%%%%%%%%%%%%%%%%%%%%%%%%%%%%%%%%%%%%%%%%%%%%%%%%%%%%%%%%%
\subsubsection{Trade time scale}\label{subsubsec:trade_time}

We use the trade sign classification in trade time scale proposed by S. Wang et
al. in \cite{Wang_2016_cross} and used in
\cite{Wang_2017,Wang_2018_copulas,Wang_2016_avg} that reads

\begin{equation}\label{eq:trade_signs_trade}
    \varepsilon^{t}\left(t,n\right)=\left\{
    \begin{array}{cc}
    \text{sgn}\left(S\left(t,n\right)-S\left(t,n-1\right)\right),
    & \text{if }\\ S\left(t,n\right) \ne S\left(t,n-1\right)\\
    \varepsilon\left(t,n-1\right),
    & \text{otherwise}
    \end{array}\right.
\end{equation}

$\varepsilon^{t}\left( t,n \right) = +1$ implies a trade triggered by a market
order to buy, and a value $\varepsilon^{t}\left( t,n \right) = -1$ indicates a
trade triggered by a market order to sell.

In the second case of the classification, if two consecutive trades with the
same trading direction did not exhaust all the available volume at the best
quote, the trades would have the same price, and in consequence they will have
the same trade sign.

With this classification we obtain trade signs for every single trade in the
data set. According to \cite{Wang_2016_cross}, the average accuracy of the
classification is $85\%$ for the trade time scale.

TAQ time step is one second, and as it is impossible to find the
correspondences between trades and midpoint prices values inside a second step,
We used the last midpoint price of every second as the representative value of
each second. This introduce an apparent shift between trade signs and returns.
In fact, we set the last midpoint price from the previous second as the first
midpoint price of the current second, as explained in \cite{Wang_2016_cross}.

As we know the second in which the trades were made, we can relate the trade
signs and the midpoint prices as shown in Fig.
% \ref{fig:relation_trades_midpoint_trade_scale}.
 For the trade time scale, they
are in general, several midpoint prices in a second. For each second we
select the last midpoint price value, and we relate it to the next second
trades. In Fig. \ref{fig:relation_trades_midpoint_trade_scale}, the last
midpoint price (circle) between the second $-1$ and $0$ is related with all the
trades (squares and triangles) in the second $0$ to $1$, and so on. It is worth
to note, in the seconds that there are no changes in the quotes, it is used the
value of the previous second (vertical line over the physical time interval).
Thus, all the seconds in the open market time have a midpoint price value, and
in consequence returns values. We assume that as long as they were not changes
in the quotes, the midpoint price remain the same as the one of the previous
second.

% \begin{figure}[htbp]
%     \centering
%     \includegraphics[width=\columnwidth]
%     {figures/02_relation_trades_quotes_trade_scale.png}
%     \caption{Sketch of data processing for trade time scale. In the midpoint
%              price time line, the vertical lines represent the change in price
%              of the quotes and the circles represent the last price change in a
%              quote in a second. In the trade signs time line, the squares
%              represent the buy market orders and the triangles represent the
%              sell market orders. The midpoint price time line and the trade
%              sign time line are shifted in one second.}
%     \label{fig:relation_trades_midpoint_trade_scale}
% \end{figure}

We computed all the analysis for the trade time scale using Equations
\ref{eq:midpoint_price_return} and \ref{eq:trade_signs_trade}.

The methodology described is an approximation to compute the response in the
trade time scale. A drawback in the computation could come from the fact that
the return of a given second is composed by the contribution of small returns
corresponding to each change in the midpoint price during a second. As we are
assuming only one value for the returns in each second, we are supposing  all
the returns in one second interval to be positive or negative with the same
magnitude, which could not be the case. This could increase or decrease the
response signal at the end of the computation.

% \begin{figure}[htbp]
%     \centering
%     \includegraphics[width=\columnwidth]{figures/02_return_contributions.png}
%     \caption{Sketch of the return contributions from every midpoint price
%              change in a second. The squares represent the rise of the price of
%              the midpoint price and the triangles represent the decrease of the
%              price of the midpoint price. We illustrate three cases: (top left)
%              the changes of the midpoint prices and return are due to the rise
%              of the prices, (top right) the changes of the midpoint prices and
%              return are due to the decrease of the prices, and (bottom) the
%              changes of the midpoint prices and return are due to a combination
%              of rise and decrease of the prices.}
%     \label{fig:return_contributions}
% \end{figure}

% Figure \ref{fig:return_contributions}
 illustrate with one example this point.
Suppose one second interval, in which they are three different midpoint prices,
and as result, three different returns for this three midpoint price
values. Furthermore, consider that the volume of limit orders that have the
corresponding midpoint prices are the same in the bid and in the ask (so the
returns have the same magnitude). In the case of the top left (top right)
sketch, all the changes are due to the rise (decrease) of the midpoint price,
that means, consumption of the best ask (bid), so all the contributions of the
individual returns in the second are positive (negative), and in consequence,
the net return is positive (negative). In the case of the bottom, the changes
are due to a combination of increase and decrease of the midpoint price, so in
the end the individual returns sum up to a net return, which can be positive or
negative, depending of the type of midpoint price values in the interval. Thus,
in this case, we are assuming at the end that all the returns were positive or
negative, what probably was not the case, and in consequence will increase or
decrease the real value of the net return.

In all the cases we choose the last change in the midpoint price in a second
interval as described before
(Fig. \ref{fig:relation_trades_midpoint_trade_scale}). We use this method
knowing that the variation in one second of the midpoint price is not large
(in average, the last midpoint price of a second differ with the average
midpoint of that second in $0.007\%$), so it can give us valuable information
about the response functions.

%%%%%%%%%%%%%%%%%%%%%%%%%%%%%%%%%%%%%%%%%%%%%%%%%%%%%%%%%%%%%%%%%%%%%%%%%%%%%%%
\subsubsection{Physical time scale}\label{subsubsec:physical_time}

We use the trade sign definition in physical time scale proposed by
S. Wang et al. in \cite{Wang_2016_cross} and used in
\cite{Wang_2017,Wang_2016_avg}, that depends on the classification in
Eq. \ref{eq:trade_signs_trade} and reads

\begin{equation}\label{eq:trade_signs_physical}
    \varepsilon^{p}\left(t\right)=\left\{
    \begin{array}{cc}
    \text{sgn}\left(\sum_{n=1}^{N\left(t\right)}\varepsilon^{t}
    \left(t,n\right)\right),
    & \text{If }N \left(t\right)>0\\
    0, & \text{If }N\left(t\right)=0
    \end{array}\right.
\end{equation}

Where $N \left(t \right)$ is the number of trades in a second interval.
$\varepsilon^{p}\left( t \right) = +1$ implies that the majority of
trades in second $t$ were triggered by a market order to buy, and a value
$\varepsilon^{p}\left( t \right) = -1$ indicates a majority of sell
market orders. In this definition, they are two ways to obtain
$\varepsilon^{p}\left( t \right) = 0$. One way is that in a particular
second there is not trades, and then no trade sign. The other  way is that the
addition of the trade signs ($+1$ and $-1$) in a second be equal to zero. In
this case, there is a balance of buy and sell market orders.

Market orders show opposite trade directions to limit order executed
simultaneously. An executed sell limit order corresponds to a buyer-initiated
market order. An executed buy limit order corresponds to a seller-initiated
market order.

As in the trade time scale, in the physical time scale we use the same strategy
to obtain the midpoint price for every second, so all the seconds in the open
market time have a midpoint price value. It is worth to note again, that even
if the second does not have a change in quotes, it will has still a midpoint
price value and a return value.

% \begin{figure}[htbp]
%     \centering
%     \includegraphics[width=\columnwidth]
%     {figures/02_relation_trades_quotes_time_scale.png}
%     \caption{Sketch of data processing for physical time scale. In the midpoint
%              price time line, the horizontal lines between seconds represent
%              the midpoint prices. In the trade signs time line, the horizontal
%              lines between seconds represent the trade sign values. The
%              midpoint price time line and the trade sign time line are shifted
%              in one second.}
%     \label{fig:relation_trades_midpoint_time_scale}
% \end{figure}

In this case we do not compare every single trade sign in a second, but the net
trade sign obtained for every second with the definition. This can be seen in
% Fig. \ref{fig:relation_trades_midpoint_time_scale},
 we related the midpoint
price of the previous second with the trade sign of the current second.

According to \cite{Wang_2016_cross}, this definition has an average
accuracy up to $82\%$ in the physical time scale.