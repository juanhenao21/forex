\section{Conclusion}\label{sec:conclusion}

We analyzed price response functions in foreign exchange markets for different
years and different time scales. We set out the trade time scale and physical
time scale to compute the price response functions for the seven major foreign
exchange pairs for three different years. Due to the characteristics of the
used data, we had to classify and sampling values to obtain the corresponding
quantities in different time scale.

The price response functions were analyzed according to the time scales. We
used trade signs and returns in trade time scale to define the price response
function. On trade time scale, the signal is noisier.
We also define the price response function using the trade signs and returns in
physical time scale. In this case, the signal is smoother. For both time scales
we observe that the signal for all the pairs increase to a maximum and then
start to slowly decrease. This behavior is also reported in correlated
financial markets. This results show the price response functions conserve
their behavior in different years and in differ markets. As the response
functions can not grow indefinitely with the time lag, they increase to a peak,
to then decrease. In both scales, the more liquid pairs have a smaller price
response function compared with the non-liquid pairs. As the liquid pairs have
more trades during the market time, the impact of each trade is reduced.
Comparing between years and scales, the price response signal is stronger in
past year than in recent years. As algorithm trading has gain great relevance,
in recent years the quantity of trades has grown, and in consequence, the
impact in the response has decreased.

Finally, we checked the spread impact in price response functions for three
different years. We divided 47 foreign exchange pairs in three groups depending
on the year average pip spread of every pair for each year. For all the year
and the time scales, the price response function signals were stronger for the
groups of pairs with larger pip spreads and weaker for the group of pairs with
smaller spreads. In all the groups in trade time scale for the three years, the
increase-maximum-decrease behavior is observed. For the physical time scale,
the behavior is not that well defined. A general average price response
behavior for each year and time scale was spotted for the groups, suggesting a
market effect on the foreign exchange pairs in each year.